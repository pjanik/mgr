\chapter{Ocena aplikacji}
\label{cha:ocena}

W tym rozdziale zaprezentowana jest kompleksowa ocena aplikacji \en podzielona
na kilka kluczowych części. Najpierw przedstawione są testy jakościowe, które
polegają na próbie odzwierciedlenia wybranych zjawisk fizycznych przy pomocy
symulatora. Z kolei testy wydajnościowe silników fizycznych w wersji
podstawowej oraz równoległej pokazują zysk jaki dało przeniesienie obliczeń na
kartę graficzną. Na podstawie tych wyników zanalizowana została dostępność
aplikacji dla potencjalnych użytkowników na różnych urządzeniach oraz ich
konfiguracjach.

\section{Modelowanie wybranych zjawisk fizycznych jako testy jakościowe symulatora}

W przypadku aplikacji edukacyjnej jaką jest symulator \en niezwykle istotne
jest aby symulowane zjawiska odzwierciedlały w sposób  wiarygodny
rzeczywistość. Z drugiej jednak strony, aplikacja musi być interaktywna i
działać w czasie rzeczywistym. Dlatego też nie można sobie pozwolić na zbyt
długi czas wykonywania, co zwykle idzie w parze z dokładnymi algorytmami i
obliczeniami. Z tego też powodu zastosowane silniki fizyczne (por. rozdział
\ref{sec:silnikiFizyczne}) balansują pomiędzy poprawnością fizyczną, a
wydajnością. Jest to dopuszczalne, ponieważ symulacja jest zorientowana
wyłącznie na aspekt wizualny.

Z powodu tego kompromisowego podejścia do pełnej dokładności obliczeń, niezwykle
istotne były testy aplikacji pod kątem podstawowej poprawności fizycznej. W tym
celu zostały przygotowane przypadki testowe, które miały za zadanie modelować
powszechnie znane zjawiska fizyczne związane z przewodnictwem cieplnym oraz
dynamiką płynów. Poniżej przedstawione są wyniki tych testów.

\subsection{Komórki Bénarda}

Modelowanie komórek Bénarda to jeden z podstawowych testów aplikacji
symulujących dynamikę płynów. Są to komórki konwekcyjne powstające w płynie
podgrzewanym od spodu. Rysunek \ref{fig:physBenard} prezentuje wyniki symulacji
przeprowadzonej przez \en.

\begin{figure}[!h]
\centering
\includegraphics[width=0.8\textwidth]{img/physics/benard}
\caption{Symulacja formowania się komórek Bénarda}
\label{fig:physBenard}
\end{figure}

Wyraźnie widać formowanie się oczekiwanych komórek. Ich układ oraz kształt
stabilizuje się po bardzo krótkim czasie symulacji. Wyniki można uznać za w
pełni satysfakcjonujące.

\subsection{Przepływ laminarny oraz turbulentny}
\label{sec:przeplywyLamTur}

Symulacja przepływów laminarnych oraz turbulentnych to następny test silnika
dynamiki płynów. Rodzaj przepływu, w przypadku gdy jest on zakłócony przez
obecność przeszkody, w głównym stopniu determinuje liczba Reynoldsa. Jest ona
zależna od:

\begin{itemize}
\item lepkości płynu,
\item prędkości przepływu,
\item średnicy przeszkody.
\end{itemize}

W związku z tym został przygotowany przypadek testowy, w którym płyn ma stałą
lepkość, a przeszkody identyczne wymiary. Zmienna jest tylko prędkość przepływu.
Dla mniejszych prędkości oczekiwanym wynikiem był przepływ laminarny, dla
większych przepływ turbulentny. Rysunek \ref{fig:physLaminarTurbulent}
prezentuje wyniki takiej symulacji przeprowadzonej przez \en.

\begin{figure}[!h]
\centering
\includegraphics[width=0.8\textwidth]{img/physics/laminarTurbulent}
\caption{Symulacja wpływu liczby Reynoldsa na rodzaj przepływu}
\label{fig:physLaminarTurbulent}
\end{figure}

Rezultaty są zgodne z oczekiwaniami. Wyraźnie widać wpływ liczby Reynoldsa
(determinowanej przez prędkość płynu) na rodzaj przepływu.

\subsection{Ścieżka wirowa von Kármána}

Ścieżka wirowa von Kármána to kolejny doskonały test silnika dynamiki płynów.
Jest to szczególny rodzaj przepływu turbulentnego, który powstaje wyłącznie dla
pewnego zakresu wartości liczby Reyonldsa (por. \ref{sec:przeplywyLamTur}).

W związku z tym został przygotowany przypadek testowy, w którym płyn ma stałą
lepkość oraz prędkość przepływu, natomiast zmienna jest tylko średnica
przeszkody. Zgodnie z powyższymi założeniami, powinno być możliwe dobranie
takich średnic przeszkód, aby wiry Kármána powstały tylko za jedną z nich.
Rysunek \ref{fig:physKarman} prezentuje wyniki takiej symulacji przeprowadzonej
przez \en.

\begin{figure}[!h]
\centering
\includegraphics[width=0.8\textwidth]{img/physics/karman}
\caption{Symulacja formowania się wirów Kármána}
\label{fig:physKarman}
\end{figure}

Rezultaty są zgodne z oczekiwaniami. Wir powstał tylko w warunkach, w których
liczba Reynoldsa była większa.

\subsection{Pojemność cieplna}

Tym razem przypadek testowy dotyczy silnika przewodnictwa cieplnego. Pojemność
cieplna to wielkość fizyczna, która charakteryzuje ilość ciepła, jaka jest
niezbędna do zmiany temperatury ciała o jednostkę temperatury. Poprawna
symulacja powinna uwzględniać ten parametr materiałów.

Aby to sprawdzić przygotowany został przypadek testowy, w którym znajdują się
dwa materiały o różnej pojemności cieplnej. Materiał o większej pojemności
powinien przewodzić ciepło znacznie lepiej niż ten o pojemności mniejszej.
Wyniki takiego eksperymentu przeprowadzonego przez symulator \en prezentuje
rysunek \ref{fig:heatCapacity}.

\begin{figure}[!h]
\centering
\includegraphics[width=0.8\textwidth]{img/physics/heatCapacity}
\caption{Symulacja wpływu pojemności cieplnej na przewodnictwo cieplne}
\label{fig:heatCapacity}
\end{figure}

Rezultaty po raz kolejny są zgodne z oczekiwaniami. Widać wyraźnie, iż po tym
samym czasie, materiał o mniejszej pojemności cieplnej nagrzał się znacznie
bardziej niż materiał o większej pojemności cieplnej.

\subsection{Aspekt edukacyjny}

Ostatni z zaprezentowanych testów jest bardziej złożony. Nie prezentuje on
jednego, konkretnego zjawiska fizycznego jak poprzednie przypadki testowe.
Skupia się on na zaprezentowaniu przykładowych możliwości symulacji oraz na
potencjalnym aspekcie edukacyjnym.

Przygotowana została scena zawierająca nieszczelnie izolowane pomieszczenie,
będące metaforą mieszkania. Jest ono ogrzewane przez element grzewczy o stałej
mocy. W jego otoczeniu został wymuszony dość mocny przepływ symulujący wiatr.
Miało to na celu pokazanie użytkownikom różnych, potencjalnych dróg ucieczki
ciepła z pomieszczeń mieszkalnych.

\begin{figure}[!h]
\centering
\includegraphics[width=0.8\textwidth]{img/physics/wind}
\caption{Symulacja ucieczki ciepła z pomieszczeń mieszkalnych}
\label{fig:wind}
\end{figure}

Rysunek \ref{fig:wind} prezentuje efekty takiej symulacji. Widoczne są dwa
wyraźne zjawiska -- utrata ciepła w okolicy nieszczelnych okien oraz utrata
ciepła związana z przewodnictwem cieplnym dachu. Pozwala to lepiej zrozumieć
użytkownikowi wpływ różnych aspektów pomieszczeń mieszkalnych (takich jak
szczelność czy izolacja) na straty ciepła. Może przyczynić się to do
rozsądniejszego i ekonomiczniejszego gospodarowania energią w jego własnym
mieszkaniu lub domu. W dobie szczególnej dbałości o środowisko naturalne
jest to niezwykle cenny i pożądany efekt edukacyjny.

\section{Testy wydajnościowe}
\label{sec:testyWydajnosciowe}

Wydajność jest kluczowym aspektem symulacji ukierunkowanej na cele edukacyjne.
Tylko odpowiednia prędkość symulacji może przyciągnąć uwagę użytkownika i
zachęcić go do dalszej eksploracji zagadnienia.

W celu osiągnięcia zadowalającej wydajności na różnych urządzeniach zastosowano
kilka technik. Podstawową jest świadoma implementacja polegająca na unikaniu
zbędnych, czasochłonnych operacji, wynikająca ze znajomości środowiska
przeglądarki internetowej. Jednak krokiem, który najbardziej przyczynił się do
powstania naprawdę wydajnej aplikacji było przeniesienie obliczeń związanych
z fizyką na procesor karty graficznej.

Niniejsza sekcja przedstawia zyski z zastosowania tej optymalizacji. Omówiona
jest także kwestia wpływu konfiguracji sprzętowej użytkownika oraz związana z
tym ogólna dostępność symulatora dla szerokiego grona odbiorców.

\subsection{Metodologia testów}
\label{sec:metodologiaTestow}

Za wyznacznik wydajności została przyjęta liczba klatek na sekundę animacji,
którą jest w stanie generować działająca aplikacja \en. Jedna klatka animacji
domyślnie składa się z:

\begin{itemize}
\item czterech kroków symulacji, 
\item odświeżenia wizualizacji.
\end{itemize}

Takie też ustawienia zostały zastosowane podczas wszystkich testów.  Aplikacja
wymusza kolejne klatki poprzez użycie metody \ow{setInterval()} z czasem 0.
Powoduje to, iż przeglądarka nie wprowadzi żadnych dodatkowych opóźnień między
klatkami i przejdzie do generowania kolejnej natychmiast, gdy będzie to
możliwe. Nie została użyta zalecana w przypadku aplikacji korzystających z
\ow{WebGL} funkcja \ow{requestAnimationFrame()}, ponieważ wprowadza ona limit
60 klatek na sekundę i koncentruje się na utrzymaniu stałego tempa animacji, a
nie na osiągnięciu maksymalnej wydajności. Zaburzyłoby to w istotny sposób
wiarygodność wyników.

Na potrzeby analizy wydajności zostały przygotowane specjalne przypadki
testowe. Różnią się one między sobą:

\begin{itemize} 
\item układem sceny, 
\item modelowanym zjawiskiem fizycznym,
\item użytymi silnikami fizycznymi:
	\begin{itemize} 
	\item wyłącznie silnik przewodnictwa cieplnego,
	\item silnik przewodnictwa cieplnego oraz silnik dynamiki płynów,
	\end{itemize}
\item rozmiarem siatki symulacyjnej.
\end{itemize}

Tabela \ref{tab:przypTest} prezentuje symboliczne nazwy przypadków testowych
wraz z ich krótką charakterystyką. Skrót HT pochodzi od ang. Heat Transfer i
oznacza, że dany przypadek testowy korzysta z silnika przewodnictwa cieplnego. Z
kolei skrót CDF pochodzi od ang. Computational Fluid Dynamics i oznacza, że dany
przypadek testowy korzysta z silnika dynamiki płynów. Wizualizację symulacji
przypadków testowych prezentują rysunki \ref{fig:ht1}, \ref{fig:ht2},
\ref{fig:cfd1} oraz \ref{fig:cfd2}.

\begin{table}[!h]
\caption{Zestawienie charakterystyki przypadków testowych do pomiaru wydajności}
\centering
\begin{tabular}{|l|c|c|c|l|}
\hline
Nazwa testu & Siatka & HT & CFD & Opis \\ \hline
\textbf{ht1-100} & $100x100$ & \checkmark & $\times$ &
symulacja przewodnictwa cieplnego \\ \hline

\textbf{ht1-512} & $512x512$ & \checkmark & $\times$ &
symulacja przewodnictwa cieplnego \\ \hline

\textbf{ht1-1024} & $1024x1024$ & \checkmark & $\times$ &
symulacja przewodnictwa cieplnego \\ \hline

\textbf{ht2-100} & $100x100$ & \checkmark & $\times$ &
symulacja przewodnictwa cieplnego \\ \hline

\textbf{ht2-512} & $512x512$ & \checkmark & $\times$ &
symulacja przewodnictwa cieplnego \\ \hline

\textbf{ht2-1024} & $1024x1024$ & \checkmark & $\times$ &
symulacja przewodnictwa cieplnego \\ 

\hline \hline 

\textbf{cfd1-100} & $100x100$ & \checkmark & \checkmark &
symulacja dynamiki płynów \\ \hline

\textbf{cfd1-256} & $256x256$ & \checkmark & \checkmark &
symulacja dynamiki płynów \\ \hline

\textbf{cfd2-100} & $100x100$ & \checkmark & \checkmark &
symulacja dynamiki płynów \\ \hline

\textbf{cfd2-256} & $256x256$ & \checkmark & \checkmark &
symulacja dynamiki płynów \\ \hline
\end{tabular}

\label{tab:przypTest}
\end{table}

\begin{figure}[!p]
\begin{minipage}[b]{0.47\linewidth}
\centering
\includegraphics[width=\textwidth]{img/perfCase/ht1}
\caption{Symulacja ht1-100/512/1024}
\label{fig:ht1}
\end{minipage}
\hspace{0.04\linewidth}
\begin{minipage}[b]{0.47\linewidth}
\centering
\includegraphics[width=\textwidth]{img/perfCase/ht2}
\caption{Symulacja ht2-100/512/1024}
\label{fig:ht2}
\end{minipage}
\end{figure}
\begin{figure}[!p]
\begin{minipage}[b]{0.47\linewidth}
\centering
\includegraphics[width=\textwidth]{img/perfCase/cfd1}
\caption{Symulacja cfd1-100/256}
\label{fig:cfd1}
\end{minipage}
\hspace{0.04\linewidth}
\begin{minipage}[b]{0.47\linewidth}
\centering
\includegraphics[width=\textwidth]{img/perfCase/cfd2}
\caption{Symulacja cfd2-100/256}
\label{fig:cfd2}
\end{minipage}
\end{figure}

\subsection{Konfiguracje sprzętowe przeznaczone do testów}
\label{sec:konfiguracjaKompow}

Testy zostały przeprowadzone na następujących komputerach:

\begin{itemize}

\item laptop Samsung QX510,
\item laptop Apple MacBook Pro, konfiguracja z roku 2010,
\item laptop Apple MacBook Pro, konfiguracja z roku 2012,
\item komputer stacjonarny, konfiguracja z roku 2008,
\item komputer stacjonarny, konfiguracja z roku 2012.

\end{itemize}

\begin{table}[!h]
\caption{Charakterystyka komputerów testowych}
\centering
\begin{tabular}{|l|l|l|l|}
\hline
Nazwa & Procesor & Karta graficzna & System operacyjny \\ \hline
Samsung QX510 & Intel Core i5 2.66GHz & NVIDIA GT 420M & Ubuntu 12.04 \\ \hline
MacBook Pro 2010 & Intel Core i7 2.66GHz & NVIDIA GT 330M & Mac OS X 10.6.8 \\ \hline
MacBook Pro 2012 & Intel Core i7 2.3GHz & NVIDIA GT 650M & Mac OS X 10.7.4 \\ \hline
Stacjonarny 2008 & Intel Core2Quad 2.5GHz & NVIDIA 9600GT & Windows 7 \\ \hline
Stacjonarny 2012 & Intel i5 3.0GHz & NVIDIA GTX 560 & Windows 7 \\ \hline
\end{tabular}
\label{tab:komputery}
\end{table}

Ich dokładną specyfikację prezentuje tabela \ref{tab:komputery}. Konfiguracje
znacznie różnią się od siebie. W zestawieniu znajdują się zarówno jednostki
mobilne, które cechują się słabszymi, energooszczędnymi podzespołami (w
szczególności dotyczy to kart graficznych), jak i komputery stacjonarne.
Ponadto, w każdej z tych grup znajdują się konfiguracje starsze, jak i
najnowsze. Komputery działają pod kontrolą trzech najpopularniejszych systemów
operacyjnych -- Linux (Ubuntu), Mac OS X oraz Windows 7. Pozwoliło to
przeprowadzić przekrojową analizę, która obrazuje rzeczywistą wydajność na
komputerach, które faktycznie mogą posiadać potencjalni użytkownicy.

\subsection{Porównanie wydajności w różnych przeglądarkach internetowych}
\label{sec:przegladarkiWyd}

Aplikacja od samego początku rozwoju była głównie testowana w przeglądarce
Google Chrome, gdyż wyraźnie było widać jej ogromną przewagę w kwestii
wydajności silnika \js. Ponadto, wg. ostatnich raportów, jest to
najpopularniejsza przeglądarka na świecie, tak więc potencjalnie najwięcej
użytkowników \en będzie z niej korzystać. Raport \cite{BrowserStats} pokazuję,
iż w lipcu 2012 roku z przeglądarki Google Chrome korzystało 42.9\%
użytkowników internetu. Drugie miejsce należało do przeglądarki Mozilla
Firefox z udziałem 33.7\%. Łącznie te dwie przeglądarki posiadają 77.6\%
,,rynku'', dlatego są traktowane priorytetowo. Kolejna popularna przeglądarka
Internet Explorer została wyłączona z testów z powodu niedostępności poza
środowiskiem Windows oraz braku wsparcia dla technologii \ow{WebGL}. Podobnie
Safari, które wspiera \ow{WebGL} jednak działa tylko na systemie operacyjnym
Mac OS. Dlatego też, ostatecznie testy wpływu przeglądarki na wydajność
symulacji zostały przeprowadzone z użyciem następujących aplikacji:

\begin{itemize}
\item Google Chrome v. 22
\item Mozilla Firefox v. 16
\item Opera v. 12.50
\end{itemize}

Użycie różnych konfiguracji sprzętowych powoduje oczywiście zmianę wyników,
jednak względne różnice w wydajności pomiędzy przeglądarkami pozostają bez
zmian. Dlatego też, w testach wydajności przeglądarek prezentowane są
wyłącznie wyniki dla laptopa Samsung QX510 (por. tablica \ref{tab:komputery}).
Wyniki dla pozostałych konfiguracji nie wniosłyby nowych oraz wartościowych
informacji.

\subsubsection{Obliczenia fizyczne wykonywane na CPU}

Wyniki pomiarów wpływu przeglądarek na wydajność symulacji w przypadku
obliczeń przeprowadzanych na CPU prezentuje tabela \ref{tab:przegladarki} oraz
wykres \ref{fig:browserPerf}. Nietrudno dostrzec ogromną przewagę przeglądarki
Google Chrome. Jest to efekt zastosowanego w niej niezwykle wydajnego silnika
\ow{JavaScript} o nazwie V8. Opera oraz Mozilla Firefox posiadają zbliżoną
wydajność, jednak Opera jest nieznacznie szybsza. Warto też zauważyć, iż kiedy
obliczenia fizyczne przeprowadzane są na CPU, jedyną przeglądarką, która
zapewnia odpowiednią prędkość wykonywania się symulacji jest Google Chrome.

\begin{table}[!h]
\caption{Wydajność symulacji na CPU w zależności od przeglądarki}
\centering
\begin{tabular}{|l|r|r|r|}
\hline
Test & Google Chrome & Mozilla Firefox & ~~~~~~~Opera \\ \hline
ht1-100 & 45.20 & 11.12 & 11.83 \\ \hline
ht1-512 & 3.24 & 0.45 & 0.98 \\ \hline
ht1-1024 & 0.92 & 0.11 & 0.25 \\ \hline
ht2-100 & 48.04 & 10.73 & 12.05 \\ \hline
ht2-512 & 3.34 & 0.40 & 0.84 \\ \hline
ht2-1024 & 0.89 & 0.11 & 0.22 \\ \hline
\hline
cfd1-100 & 22.25 & 2.51 & 3.84 \\ \hline
cfd1-256 & 4.24 & 0.38 & 0.93 \\ \hline
cfd2-100 & 20.60 & 2.35 & 4.58 \\ \hline
cfd2-256 & 3.71 & 0.30 & 0.73 \\ \hline
\end{tabular}
\label{tab:przegladarki}
\end{table}

\begin{figure}[!h]
\centering
\includegraphics[width=\textwidth]{img/browserPerf}
\caption{Wydajność symulacji na CPU w zależności od przeglądarki}
\label{fig:browserPerf}
\end{figure}

\clearpage

\subsubsection{Obliczenia fizyczne wykonywane na GPU}

W przypadku symulacji, która obliczenia fizyczne przeprowadza na GPU, nie
została przetestowana Opera.  W dniu kiedy testy były przeprowadzane, Opera
oferowała wyłącznie eksperymentalne wsparcie dla technologii \ow{WebGL} i nie
oferowała dostępu do niezbędnego rozszerzenia \ow{OES\_texture\_float}. Wyniki
pomiarów wpływu pozostałych przeglądarek na wydajność symulacji w przypadku
obliczeń przeprowadzanych na GPU prezentuje tabela \ref{tab:przegladarkiGPU}
oraz wykres \ref{fig:browserPerfGPU}.

\begin{table}[H]
\caption{Wydajność symulacji na GPU w zależności od przeglądarki}
\centering
\begin{tabular}{|l|r|r|}
\hline
Test & Google Chrome & Mozilla Firefox \\ \hline
ht1-100 & 51.57 & 37.33 \\ \hline
ht1-512 & 12.94 & 12.99 \\ \hline
ht1-1024 & 3.73 & 3.64 \\ \hline
ht2-100 & 48.84 & 37.59 \\ \hline
ht2-512 & 12.52 & 12.28 \\ \hline
ht2-1024 & 3.60 & 3.47 \\ \hline
\hline
cfd1-100 & 36.67 & 30.59 \\ \hline
cfd1-256 & 13.47 & 12.35 \\ \hline
cfd2-100 & 33.28 & 31.01 \\ \hline
cfd2-256 & 10.85 & 11.18 \\ \hline
\end{tabular}
\label{tab:przegladarkiGPU}
\end{table}

\begin{figure}[H]
\centering
\includegraphics[width=0.9\textwidth]{img/browserPerfGPU}
\caption{Wydajność symulacji na GPU w zależności od przeglądarki}
\label{fig:browserPerfGPU}
\end{figure}

W przypadku obliczeń na GPU sytuacja diametralnie się zmienia. Różnice między
przeglądarkami, tak istotne przy obliczeniach na CPU, stają się znikome.
Wynika to z tego, iż najbardziej obciążające obliczenia przeniesione są na
kartę graficzną i to jej wydajność jest decydująca, a nie wydajność samego
silnika \ow{JavaScript}.  Potwierdza to fakt, iż wraz ze wzrostem wielkości
siatki symulacyjnej, różnice między przeglądarkami zanikają. Przewagę Google
Chrome widać wyłącznie dla testów działających na najmniejszych siatkach. Są
to jedyne przypadki, gdzie czas potrzeby na realizacje instrukcji \js
stanowi zauważalny procent czasu wykonywania się całej symulacji.

Innym ważnym wnioskiem jest to, iż użycie technologii \ow{WebGL} pozwala
zniwelować różnice między wydajnością przeglądarek. O ile w przypadku obliczeń
na CPU jedynym rozsądnym środowiskiem wykonywania się symulacji była
przeglądarka Google Chrome, to przy zastosowaniu \ow{WebGL} możliwe staje się
również użycie Mozilli Firefox oraz każdej innej przeglądarki, która wspiera
\ow{WebGL} (np. Safari czy też w niedalekiej przyszłości także Opery).

\subsection{Analiza zysku wydajności wynikający z przeniesienia obliczeń na GPU}
\label{sec:analizaGPUCPU}

Kluczową optymalizacją wydajności symulatora \en było przeniesienie obliczeń
związanych z fizyką na procesor karty graficznej przy wykorzystaniu technologii
\ow{WebGL}. Szczegóły implementacyjne oraz najważniejsze zagadnienie z tym
związane porusza rozdział \ref{cha:oblGPU}. Niniejsza sekcja przedstawia i
analizuje efekty jakie przyniosła ta optymalizacja.

\begin{description}

\item[Metodologię testów] definiuje podrozdział \ref{sec:metodologiaTestow}.

\item[Konfiguracje sprzętowe] użyte podczas testów przybliża podrozdział
\ref{sec:konfiguracjaKompow}.

\item[Wybór przeglądarki internetowej] do testów jest niezwykle istotnym
czynnikiem, który w największym stopniu determinuje ocenę optymalizacji. Wynika
to z bardzo dużych różnic między wydajnością interpreterów \ow{JavaScript} w
różnych przeglądarkach (por. podrozdział \ref{sec:przegladarkiWyd}). Oczywistym
jest, że przeglądarka o niższej wydajności silnika \ow{JavaScript} zanotuje
znacznie większe przyspieszenie niż przeglądarka o silniku bardzo wydajnym.
Dlatego też do testów została użyta \textbf{najwydajniejsza} z dostępnych
przeglądarek -- Google Chrome. Dzięki temu testy przedstawiają
\textbf{minimalne, pesymistyczne} oczekiwane przyspieszenie związane z
przeniesieniem obliczeń  na GPU. Jednocześnie należy podkreślić, iż na mniej
wydajnych przeglądarkach optymalizacja przynosi jeszcze więcej korzyści i
często staje się niezbędna aby symulację można było określić mianem
interaktywnej.

\end{description}

Wyniki testów na różnych komputerach przestawiają tabele od
\ref{tab:wynikiWebGL_first} do \ref{tab:wynikiWebGL_last}. Zgodnie z przyjętą
metodologią, miarą wydajności jest liczba klatek na sekundę jaką jest w stanie
wygenerować przeglądarka podczas działania symulacji. Kolumna \ow{CPU}
prezentuje wydajność symulacji na procesorze głównym, a kolumna \ow{GPU} na
procesorze karty graficznej. Z kolei kolumna \ow{GPU/CPU} prezentuje stosunek
wartości z kolumny \ow{GPU} do wartości z kolumny \ow{CPU}. Jest to pomocniczy
współczynnik, który bardzo dobrze charakteryzuje i wizualizuje zysk z
optymalizacji.

\begin{table}[!htp]
\caption{Wydajność symulacji na CPU oraz GPU -- Google Chrome, laptop Samsung QX510}
\centering
\begin{tabular}{|l|r|r|>{\bfseries}c|}
\hline
\cellcolor{t} Test & \cellcolor{cpu} CPU & \cellcolor{gpu} GPU & \cellcolor{gc} GPU/CPU \\ \hline
ht1-100 & 41.98 & 68.73 & 1.64 \\ \hline
ht1-512 & 3.15 & 12.30 & 3.90 \\ \hline
ht1-1024 & 0.71 & 3.58 & 5.04 \\ \hline
ht2-100 & 51.37 & 62.36 & 1.21 \\ \hline
ht2-512 & 2.86 & 12.61 & 4.41 \\ \hline
ht2-1024 & 0.80 & 3.75 & 4.69 \\ \hline
\hline
cfd1-100 & 18.84 & 43.76 & 2.32 \\ \hline
cfd1-256 & 3.88 & 13.56 & 3.49 \\ \hline
cfd2-100 & 20.60 & 43.28 & 1.62 \\ \hline
cfd2-256 & 3.71 & 10.85 & 2.92 \\ \hline
\end{tabular}
\label{tab:wynikiWebGL_first}
\end{table}

\begin{table}[!htp]
\caption{Wydajność symulacji na CPU oraz GPU -- Google Chrome, laptop Apple MacBook Pro 2010}
\centering
\begin{tabular}{|l|r|r|>{\bfseries}c|}
\hline
\cellcolor{t} Test & \cellcolor{cpu} CPU & \cellcolor{gpu} GPU & \cellcolor{gc} GPU/CPU \\ \hline
ht1-100 & 43.10 & 92.60 & 2.15 \\ \hline
ht1-512 & 3.20 & 15.50 & 4.84 \\ \hline
ht1-1024 & 0.91 & 4.90 & 5.38 \\ \hline
ht2-100 & 43.70 & 92.10 & 2.11 \\ \hline
ht2-512 & 2.91 & 15.50 & 5.33 \\ \hline
ht2-1024 & 0.73 & 4.55 & 6.23 \\ \hline
\hline
cfd1-100 & 21.00 & 55.10 & 2.62 \\ \hline
cfd1-256 & 2.30 & 14.60 & 6.35 \\ \hline
cfd2-100 & 8.00 & 48.00 & 6.00 \\ \hline
cfd2-256 & 1.70 & 12.20 & 7.18 \\ \hline
\end{tabular}
\label{tab:wynikiWebGL_2}
\end{table}

\begin{table}[!htp]
\caption{Wydajność symulacji na CPU oraz GPU -- Google Chrome, laptop Apple MacBook Pro 2012}
\centering
\begin{tabular}{|l|r|r|>{\bfseries}c|}
\hline
\cellcolor{t} Test & \cellcolor{cpu} CPU & \cellcolor{gpu} GPU & \cellcolor{gc} GPU/CPU \\ \hline
ht1-100 & 57.00 & 116.00 & 2.04 \\ \hline
ht1-512 & 4.10 & 53.60 & 13.07 \\ \hline
ht1-1024 & 1.10 & 14.90 & 13.55 \\ \hline
ht2-100 & 79.80 & 118.80 & 1.49 \\ \hline
ht2-512 & 4.60 & 50.30 & 10.93 \\ \hline
ht2-1024 & 0.96 & 14.20 & 14.79 \\ \hline
\hline
cfd1-100 & 17.10 & 101.00 & 5.91 \\ \hline
cfd1-256 & 2.80 & 36.60 & 13.07 \\ \hline
cfd2-100 & 19.90 & 93.00 & 4.67 \\ \hline
cfd2-256 & 2.80 & 36.60 & 13.07 \\ \hline
\end{tabular}
\label{tab:wynikiWebGL_3}
\end{table}

\begin{table}[!htp]
\caption{Wydajność symulacji na CPU oraz GPU -- Google Chrome, komputer stacjonarny 2008}
\centering
\begin{tabular}{|l|r|r|>{\bfseries}c|}
\hline
\cellcolor{t} Test & \cellcolor{cpu} CPU & \cellcolor{gpu} GPU & \cellcolor{gc} GPU/CPU \\ \hline
ht1-100 & 35.00 & 78.00 & 2.23 \\ \hline
ht1-512 & 2.50 & 37.00 & 14.80 \\ \hline
ht1-1024 & 0.69 & 9.00 & 13.04 \\ \hline
ht2-100 & 39.00 & 83.00 & 2.13 \\ \hline
ht2-512 & 2.25 & 37.00 & 16.44 \\ \hline
ht2-1024 & 0.65 & 10.60 & 16.31 \\ \hline
\hline
cfd1-100 & 16.55 & 64.10 & 3.87 \\ \hline
cfd1-256 & 2.50 & 30.00 & 12.00 \\ \hline
cfd2-100 & 12.80 & 65.20 & 5.09 \\ \hline
cfd2-256 & 2.15 & 26.30 & 12.23 \\ \hline
\end{tabular}
\label{tab:wynikiWebGL_4}
\end{table}

\clearpage

\begin{table}[!htp]
\caption{Wydajność symulacji na CPU oraz GPU -- Google Chrome, komputer stacjonarny 2012}
\centering
\begin{tabular}{|l|r|r|>{\bfseries}c|}
\hline
\cellcolor{t} Test & \cellcolor{cpu} CPU & \cellcolor{gpu} GPU & \cellcolor{gc} GPU/CPU \\ \hline
ht1-100 & 65.00 & 149.00 & 2.29 \\ \hline
ht1-512 & 4.50 & 67.00 & 14.89 \\ \hline
ht1-1024 & 1.25 & 21.00 & 16.80 \\ \hline
ht2-100 & 79.00 & 149.00 & 1.89 \\ \hline
ht2-512 & 5.20 & 69.00 & 13.27 \\ \hline
ht2-1024 & 1.20 & 22.00 & 18.33 \\ \hline
\hline
cfd1-100 & 30.45 & 140.00 & 4.60 \\ \hline
cfd1-256 & 5.80 & 58.00 & 10.00 \\ \hline
cfd2-100 & 21.60 & 143.00 & 6.62 \\ \hline
cfd2-256 & 4.65 & 48.00 & 10.32 \\ \hline
\end{tabular}
\label{tab:wynikiWebGL_last}
\end{table}

Bardzo wyraźna jest zależność wyników od konfiguracji sprzętowej. Wyniki można
rozpatrywać pod kątem wartości bezwzględnych oraz pod kątem zysku z zastosowania
optymalizacji polegającej na przeniesieniu obliczeń na GPU (reprezentowanego
przez współczynnik GPU/CPU). W przypadku wartości bezwzględnych, oczywistym
jest, iż najszybsza powinna okazać się najmocniejsza jednostka w zestawieniu. W
przypadku powyższych testów nie było niespodzianki, zdecydowanie najszybszy
okazał się komputer stacjonarny z roku 2012, co znajduje uzasadnienie w jego
parametrach.

W przypadku znacznie bardziej interesującego parametru, jakim jest stosunek
wydajności GPU do CPU, można zauważyć dwie grupy charakteryzujące się dużymi
różnicami w wynikach. Zdecydowanie większy zysk z przeniesienia obliczeń na GPU
odnotowują oba komputery stacjonarne oraz laptop Apple MacBook z roku 2012.
Pozostałe dwa laptopy prezentują trochę mniej imponujące rezultaty tej
optymalizacji. Tłumaczy to przede wszystkim zestawienie kart graficznych oraz
ich odniesienie do mocy procesora danej konfiguracji. Dwie jednostki, które
osiągają najmniejszy zysk, posiadają stosunkowo wydajne procesory, natomiast
karty graficzne są zdecydowanie nastawione na energooszczędność, a nie
maksymalną wydajność. Ponadto, w obecnych czasach kolejne modele kart
graficznych wypuszczanych na rynek przynoszą znacznie większe zmiany w mocy
obliczeniowej niż kolejne modele procesorów głównych. Dlatego też, nowsze karty
graficzne w zestawieniu prezentują znacznie bardziej imponujące rezultaty.

Powyższą obserwację można uogólnić, stwierdzając iż komputery nowsze, o większej
mocy obliczeniowej, korzystają ze zrównoleglenia obliczeń znacznie bardziej niż
komputery generalnie mniej wydajne i starsze. Jest to obiecująca tendencja, gdyż
prognozuje coraz większe korzyści płynące z przenoszenia obliczeń ogólnego
zastosowania na procesory kart graficznych.

Zdecydowanie największy wpływ na przyrost wydajność podczas obliczeń
równoległych ma rozmiar siatki symulacyjnej. Relacje współczynnika GPU/CPU oraz
rozmiaru siatki prezentuje tabela \ref{tab:gpucpu} oraz odpowiadający jej wykres
\ref{fig:gpucpu}. Dane uśredniono z testów na wszystkich pięciu konfiguracjach
sprzętowych.

\begin{table}[!htp]
\caption{Stosunek wydajności obliczeń na GPU do CPU w zależności od rozmiaru
siatki symulacyjnej}
\centering
\begin{tabular}{|l|>{\bfseries}c|}
\hline
\cellcolor{t} Siatka & \cellcolor{gc} GPU/CPU \\ \hline
100x100 & 3.12 \\ \hline
256x256	& 9.06 \\ \hline
512x512	& 10.19 \\ \hline
1024x1024 & 11.42 \\ \hline
\end{tabular}
\label{tab:gpucpu}
\end{table}

\begin{figure}[!h]
\centering
\includegraphics[width=\textwidth]{img/gpucpu}
\caption{Stosunek wydajności obliczeń na GPU do CPU w zależności od rozmiaru
siatki symulacyjnej}
\label{fig:gpucpu}
\end{figure}

Wyniki udowadniają wartość zastosowanej optymalizacji. Już dla najmniejszej
siatki, średni wzrost wydajności jest ponad \textbf{trzykrotny}. Dla większych
siatek wydajność jest średnio \textbf{dziesięć razy większa} niż podczas
obliczeń na CPU.

Odpowiednio gęsta siatka pokazuje potencjał jednostek graficznych. Ponadto,
wzrost zysku z optymalizacji wraz ze wzrostem rozmiaru siatki wiążę się też
zapewne z niwelowaniem znaczenia instrukcji w języku \ow{JavaScript} wspólnych
dla wersji zarówno sekwencyjnej jak i równoległej. Gdy rozmiar siatki jest
odpowiednio duży, procent czasu poświęcony na wykonywanie tych instrukcje staje
się pomijalnie mały.

Warto też przypomnieć, że zgodnie z przyjętymi założeniami, są to wyniki dla
przeglądarki Google Chrome, która posiada najszybszy silnik \js, a więc
\textbf{najmniej korzystny} do testowania zysków z przeniesienia obliczeń na
GPU. W przypadku Mozilli Firefox, już dla najmniejszych siatek symulacja na GPU
działa około \textbf{dwanaście} razy szybciej niż symulacja na CPU, a dla
większych siatek około \textbf{trzydzieści} razy (por. tabele
\ref{tab:przegladarki} oraz \ref{tab:przegladarkiGPU}).

\subsection{Wpływ przeniesienia obliczeń na GPU na jakość symulacji}

Oczywiście zrównoleglenie symulacji przekłada się na jej szybsze wykonywanie.
Dokładne wyniki przedstawia poprzednia sekcja \ref{sec:analizaGPUCPU}. W
kontekście aplikacji edukacyjnej jest to bardzo ważne -- symulator staje się
bardziej ,,interaktywny'', a oczekiwane rezultaty widać dużo szybciej.
Doświadczenia i analizy pokazują, że zbyt wolne działanie aplikacji bardzo
często skutkuje szybkim zniechęceniem i znużeniem użytkownika.

Dzięki przeniesieniu obliczeń na GPU, symulacja przyspieszyła około trzykrotnie
już dla siatek 100x100. Często tylko zastosowanie tej optymalizacji pozwala, aby
animacja była odtwarzana z prędkością większą niż 25--30 klatek na sekundę. Jest
to graniczna wartość, dla której ludzkie oko interpretuje animację jako płynną.
Dlatego też zysk z optymalizacji jest nie do przecenienia.

Użycie większej siatki symulacyjnej skutkuje jeszcze większym wzrostem
wydajności podczas obliczeń równoległych. Wydawać by się mogło, iż większa
siatka powinna przynieść automatycznie lepszą jakość symulacji. Jednak przy
zastosowanych algorytmach fizycznych, większa siatka wprowadza też pewne
problemy -- trudniej jest symulować wybrane zjawiska fizyczne. Potrzebne jest
znacznie dokładniejsze rozwiązywanie układów równań liniowych (czyli w praktyce
wykonanie większej liczby kroków w metodzie relaksacyjnej), co powoduje
zwielokrotnienie czasu potrzebnego na obliczenia. Efekt ten przedstawiają
rysunki \ref{fig:relaxGPU1} oraz \ref{fig:relaxGPU2}. Przy obecnej wydajności
procesorów głównych oraz graficznych, powoduje to iż używanie gęstych siatek
symulacji staje się niepraktyczne. Ponadto, nowy efekt wizualny nie jest wart
tak znacznego spadku wydajności. Dlatego też większość przykładów w finalnej
wersji aplikacji korzysta z siatek 100x100. Taki rozmiar zapewnia dobrą jakość
wizualną i merytoryczną symulacji, a dzięki przeniesieniu obliczeń na GPU
również doskonałą płynność działania.

\begin{figure}[!h]
\centering
\includegraphics[width=0.6\textwidth]{img/relaxGPU1}
\caption{Niewłaściwie formowanie się ścieżki wirowej von Karmana (obliczenia na
GPU, siatka 256x256, 10 kroków relaksacji)}
\label{fig:relaxGPU1}
\end{figure}

\begin{figure}[!h]
\centering
\includegraphics[width=0.6\textwidth]{img/relaxGPU2}
\caption{Właściwe formowanie się ścieżki wirowej von Karmana (obliczenia na GPU,
siatka 256x256, 60 kroków relaksacji)}
\label{fig:relaxGPU2}
\end{figure}

\clearpage

\section{Ocena dostępności aplikacji dla potencjalnych użytkowników}

Symulator \en do działania wymaga wyłącznie przeglądarki internetowej. W
dzisiejszych czasach jest to standardowa aplikacja znajdująca się na prawie
każdym komputerze. Do uruchomienia symulatora \en nie jest wymagana żadna
instalacja zewnętrznych bibliotek, rozszerzeń, czy też uprawnienia
administratora. Dlatego też potencjalnemu użytkownikowi wystarczy komputer
średniej klasy z zainstalowaną przeglądarką internetową oraz dostępem do
internetu. Biorąc pod uwagę możliwości aplikacji, są to niewątpliwie niskie
wymagania.

Wpływ posiadanej konfiguracji sprzętowej i oprogramowania na działanie
symulatora jest znaczny. Gdy nie jest możliwe korzystanie z akceleracji na
karcie graficznej, wydajność silnika \ow{JavaScript} zastosowanego w
przeglądarce internetowe ma decydujące znacznie. Testy przedstawia sekcja
\ref{sec:przegladarkiWyd}. Najlepszym wyborem w tym momencie jest przeglądarka
Google Chrome, jednak w związku z dynamicznym rozwojem wszystkich wiodących
przeglądarek, sytuacja ta może się zmienić w niedalekiej przyszłości.

Jednak problem zależności od konkretnej przeglądarki likwiduje przeniesienie
obliczeń na kartę graficzną. Interpreter \ow{JavaScript} traci na znaczeniu,
jednak wymagane jest wsparcie przeglądarki dla technologii \ow{WebGL}. W tym
momencie zapewnia je Google Chrome, Mozilla Firefox, Safari oraz Opera (jeszcze
w fazie eksperymentalnej). Czyli przeglądarki, które posiadają zdecydowaną
większość rynku. Ponadto, wsparcie dla \ow{WebGL} w niedalekiej przyszłości
powinna zapewniać każda, licząca się przeglądarka internetowa.

Przy użyciu tej optymalizacji, ciężar obliczeń spada na kartę graficzną. Jednak
nawet niewydajne, energooszczędne karty graficzne przeznaczone do laptopów radzą
sobie z tym zdaniem doskonale. Dlatego też nie powinno być to problemem dla
użytkownika. Uśredniając zgromadzone wyniki testów (por
\ref{sec:analizaGPUCPU}), karty graficzne osiągnęły od trzech do dziesięciu razy
lepszą wydajność niż procesory główne i zapewniły doskonałą płynność symulacji.

