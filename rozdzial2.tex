\chapter{Przeniesienie obliczeń fizycznych na procesor karty graficznej}

Niniejszy rozdział prezentuje techniki zastosowane w celu przeniesienia
głównych obliczeń fizycznych na kartę graficzną. Na początku opisane są
tradycyjne podejścia do tego problemu dla aplikacji działających w natywnym
środowisku systemu operacyjnego oraz ich odniesienie do środowiska oferowanego
przez przeglądarki internetowe. Następnie zaprezentowana jest technologia
\mbox{WebGL} dzięki której dokonano zrównoleglenia obliczeń w przypadku
symulatora fizycznego będącego przedmiotem opracowania niniejszej pracy. W
ostatnim podrozdziale opisane zostały najważniejsze szczegóły implementacyjne,
które mogą być niezwykle pomocne przy próbach podobnej optymalizacji innych
aplikacji.

Dzięki przeniesieniu obliczeń na GPU uzyskano niezwykle istotny wzrost
wydajności. Dokładna analiza zysków ze zrównoleglenia symulacji jest tematyką
kolejnego rozdziału [TODO: podać dokładny rozdział].

\section{Typowe metody przenoszenia obliczeń na GPU, a przeglądarka internetowa}

Współczesna karty graficzne posiadają ogromną moc obliczeniową -- wielokrotnie
większą od centralnego procesora przy założeniu, że obliczenia da się
wykonywać w sposób równoległy. Pomysł, aby przenieść część obliczeń ogólnego
zastosowania na kartę graficzną pojawił się wraz z dynamicznym rozwojem
procesorów graficznych. Szczególnie istotnym momentem było wprowadzenie
programowalnych jednostek cieniujących (specyfikacja \emph{DirectX 8}). Dalszy
rozwój obliczeń ogólnego zastosowania na kartach graficznych (ang. \emph
{General-purpose computing on graphics processing units}, w skrócie GPGPU)
miał miejsce wraz z wprowadzeniem technologii, które ukryły złożoność dostępu
do zasobów karty graficznej i udostępniły interfejs wysokiego poziomu.
Wiodącymi technologiami tego typu są \emph{OpenCL} (rozwiązanie otwarte) oraz
\emph{CUDA} (zamknięte rozwiązanie firmy NVIDIA, działające wyłącznie na
sprzęcie tego producenta).

\subsection{Niskopoziomowe programowanie jednostek cieniujących}

Jest to najstarsze podejście do przeprowadzania obliczeń na procesorze karty
graficznej. Programista w swoisty sposób ,,oszukuje'' kartę graficzną,
przeprowadzając renderowanie prostej geometrii wyłącznie w celu uruchomienia
własnych programów jednostek cieniujących, które wykonują obliczenia często
nie mające nic wspólnego z generowaniem obrazu.

Programowanie tego typu bywa wymagające. Łatwo popełnić błędy, a programista
musi mieć przynajmniej podstawową wiedzę o programowaniu grafiki 3D.

\subsection{Technologie wyższego poziomu}

Technologie, które przyczyniły się do gwałtownego wzrostu popularności
obliczeń na kartach graficznych to szczególnie \emph{OpenCL} oraz \emph{CUDA}.
Udostępniają  one znacznie wyższy poziom abstrakcji -- programista jest
zwolniony z obowiązku przyswojenia sobie niskopoziomowych mechanizmów
rządzących działaniem kart graficznych (choć ta wiedza pozwala tworzyć
aplikacje efektywniejsze).

\subsection{Środowisko przeglądarka internetowa}

Opisane wcześniej technologie dotyczą aplikacji pisanych w natywnym środowisku
systemu operacyjnego. Aby programować jednostki cieniujące wystarczy
podstawowy dostęp do standardowego interfejsu OpenGL bądź Direct3D.
Implementacje tych interfejsów można znaleźć dla prawie każdego współczesnego
języka programowania. Technologie wyższego poziomu takie jak OpenCL oraz CUDA
również posiadają implementacje w wielu różnych językach, choć najczęstszym
środowiskiem ich działania są zwykle aplikacje napisane w C bądź C++.

Przeglądarka internetowa udostępnia programiście JavaScript środowisko bardzo
ograniczone ze względów bezpieczeństwa. Technologie programowania kart
graficznych wyższego poziomu nie są (jeszcze) dostępne. Aktualnie trwają
intensywne prace nad implementacją standardu OpenCL w przeglądarce
internetowej - WebCL. Jednak technologia ta w aktualnym momencie jest na
bardzo wczesnym etapie rozwoju. Więcej informacji można znaleźć na stronie
internetowej: http://www.khronos.org/webcl/.

Jednak niedawno, wraz z nadejściem standardu HTML5, został dodany podstawowy
dostęp do zasobów karty graficznej. Pojawiła się możliwość programowania
jednostek cieniujących kart graficznych i tym samym przeprowadzania na nich
obliczeń ogólnego zastosowania. Zadania te są realizowane przy użyciu
technologii WebGL, która jest implementacją standardu OpenGL ES 2.0.

\section{WebGL -- otwarcie dostępu do zasobów karty graficznej w przeglądarce}

\section{Opis najważniejszych zagadnień implementacji silników fizycznych przy użyciu WebGL}


