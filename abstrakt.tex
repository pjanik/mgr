\begin{abstract}

Celem niniejszej pracy było stworzenie wydajnej symulacji dynamiki płynów oraz
przewodnictwa cieplnego działającej w środowisku przeglądarki internetowej.
Aplikacja ma charakter edukacyjny, a jej głównym zadaniem jest wspieranie
użytkownika w dogłębnym zrozumieniu procesu transferu energii, w tym zjawisk
takich jak przewodnictwo cieplne, konwekcja czy różnorodne przepływy gazów i
cieczy. Projekt został zrealizowany przy współpracy \mbox{z The Concord
Consortium}, amerykańską organizacją \mbox{non-profit} zajmującą się wspieraniem
edukacji poprzez technologię.

Implementacja symulatora w przeglądarce internetowej była niezwykle istotna ze
względu na jego edukacyjne zastosowanie -- kluczowym wymaganiem była dostępność
dla jak najszerszego grona użytkowników, przenośność, wieloplatformowość oraz
możliwość łatwego osadzania w wirtualnych podręcznikach szkolnych nowej
generacji. Z drugiej strony, o jakości symulacji fizycznej w dużej mierze
decyduje jej wydajność czyli efektywne wykorzystanie zasobów. Do niedawna stało
to w sprzeczności z implementacją w języku \emph{JavaScript} oraz wykonywaniem w
środowisku przeglądarki internetowej. 
 
Realizacja tych pozornie wykluczających się wymagań została osiągnięta dzięki
implementacji uwzględniającej budowę i ograniczenia silników \emph{JavaScript} w
nowoczesnych przeglądarkach oraz dzięki przeniesieniu większości obliczeń na
procesor karty graficznej wykorzystując technologię \emph{WebGL}. Jest to
podejście nowatorskie, gdyż dopiero wraz z niedawnym rozpowszechnieniem się
standardu \emph{HTML5} zasoby kart graficznych stały się dostępne dla aplikacji
internetowych w tak szerokim zakresie.

W pracy zostały przedstawione najważniejsze, nowoczesne techniki umożliwiające
tworzenie wydajnych, równoległych aplikacji działających w przeglądarce
internetowej z wykorzystaniem języka \emph{JavaScript} oraz technologii
\emph{WebGL}. Szczegółowo zostały również opracowane rezultaty oraz korzyści
płynące ze zrównoleglenia symulacji fizycznej. W~\mbox{efekcie} powstał wydajny
symulator, który dzięki swojej dostępności oraz jakości może mieć niezwykle
szeroki wpływ na edukację i zrozumienie fizyki przez użytkowników na różnych
etapach procesu kształcenia.

\end{abstract}
