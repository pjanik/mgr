\chapter{Wprowadzenie}
\label{cha:wprowadzenie}

Większość ludzi w swoich domach i pracy może korzystać z dorobku technologicznej
rewolucji, która miała miejsce w ostatnich latach. Jednak w niektórych
dziedzinach życia zmiany następują znacznie wolniej - jedną z nich jest
edukacja. Komputery i internet stały się dostępne w większości szkół, jednak
programy i metody nauczania często nie nadążają za postępem technologicznym, są
wciąż reliktem poprzedniej epoki. Uczniowie pracują na komputerach najnowszej
generacji, jednak zwykle wykorzystują tylko ułamki ich możliwości, nie mając
dostępu do narzędzi, które faktycznie mogłyby przyczynić się do lepszego
zrozumienia poruszanych na lekcjach zagadnień. Najczęściej komputer i internet
stają się po prostu źródłem łatwo dostępnej wiedzy czy też miejscem gdzie pewne
problemy można próbować rozwiązywać wspólnie. Jest to oczywiście prawidłowe i
wartościowe wykorzystanie nowoczesnej technologi, ale jednocześnie też dosyć
powierzchowne i~niewyczerpujące jej pełnych możliwości. Problemy i wyzwania
stojące przed współczesną edukacją szerzej porusza Andrew A. Zucker
\cite{Zuc2009}.

Edukacja może skorzystać na technologicznej rewolucji w znacznie większym
stopniu -- jednym z pomysłów są wirtualne laboratoria, które pozwolą uczniom
eksplorować wybrane zagadnienia w sposób interaktywny, szczególnie podczas
nauczania przedmiotów ścisłych i przyrodniczych, tak istotnych w dzisiejszych
czasach. Cyfryzacja powinna zmienić tradycyjne oblicze lekcji z podręcznikiem i
zeszytem na pracę przy narzędziach edukacyjnych nowej generacji, wykorzystując
powszechną dostępność nowoczesnych technologii. Takie aplikacje również
doskonale wpisują się w popularną ideę cyfrowych podręczników. Dosłowne
przeniesienie zawartości papierowych książek na ekrany komputerów nie wiązałoby
się ze znaczącymi zmianami - inny byłby tylko nośnik słów, wiedzy. Bez zmian
natomiast pozostałby sam proces i metody uczenia przez uczniów i studentów.
Jednak jeśli wirtualny podręcznik zostanie zintegrowany z interaktywnymi
aplikacjami, pozwoli to zupełnie zmienić oblicze nauki. Uczeń będzie miał
możliwość prawdziwej eksploracji zagadnień, eksperymentowania we własnym domu,
przed własnym komputerem, podczas codziennej nauki, która może zamienić się w
prawdziwą, wartościową i przede wszystkim rozwijającą przygodę.

Najlepszym pomysłem dla podręczników przyszłości wydaje się umiejscowienie ich w
internecie. Dzięki dystrybucji poprzez to medium można uzyskać niezwykle łatwy i
powszechny dostęp, jako że połączenie z internetem jest w dzisiejszych czasach
czymś w pełni osiągalnym. Internetowa dystrybucja niesie również mnóstwo
korzyści nie tylko dla użytkowników podręczników, ale także dla ich twórców -
wystarczy wymienić zalety takie jak łatwość aktualizacji i docierania do
odbiorców. W związku z tym, również narzędzia stanowiące interaktywne elementy
podręczników przyszłości powinny być przystosowane do działania w środowisku
przeglądarki internetowej. Jest to zadanie wymagające, jednak niedawny rozwój
technologii i standardów internetowych takich jak HTML5 oraz WebGL, jak również
gwałtowne zmiany w samych przeglądarkach internetowych, dają ogromne możliwości
w tej materii.

Wymienione pomysły nie są tylko planami na przyszłość - te zmiany już powoli
następują, cyfrowe podręczniki i wirtualne laboratoria są trakcie rozwoju. Jedną
z organizacji zajmujących się wprowadzaniem najnowszych osiągnięć techniki do
szkół jest \mbox{The Concord Consortium}.

\section{Cel pracy}
\label{sec:celPracy}

W wyniku współpracy ze wspomnianą organizacją The Concord Consortium powstał
wydajny symulator fizyczny o nazwie \en. Prezentuje on zjawisko przewodnictwa
cieplnego oraz dynamikę płynów, a jego środowiskiem działania jest
przeglądarka internetowa. Ta interaktywna aplikacja doskonale wpisuje się w
przedstawioną ideę wirtualnych podręczników i laboratoriów, umożliwiając
użytkownikom łatwiejsze zrozumienie praw fizyki, które rządzą transferem
energii.

Celem niniejszej pracy jest przedstawienie rozwiązań, które umożliwiły powstanie
symulatora, ze szczególnym naciskiem na technologię WebGL, której niestandardowe
i nowatorskie zastosowanie pozwoliło zrównoleglić obliczenia fizyczne i tym
samym osiągnąć znaczący wzrost wydajności. 

\section{Organizacja dokumentu}
\label{sec:organizacjaDokumentu}

Dalsze rozdziały przedstawiają kolejno:

\begin{itemize} \item Przedstawienie możliwości symulatora \en, wprowadzenie
do problematyki symulacji dynamiki płynów i przewodnictwa cieplnego oraz
przegląd istniejących, podobnych rozwiązań.

\item Opis podstawowej implementacji aplikacji, ze szczególnym uwzględnieniem
zagadnień związanych z jej architekturą i wnioskami, które można rozszerzyć na
ogół złożonych systemów \ow{JavaScript}.

\item Prezentację kluczowych technik dzięki którym udało się zrównoleglić
silniki fizyczne symulatora \en przy pomocy technologii \ow{WebGL}.

\item Ocenę systemu, w szczególności testy jakościowe, wydajnościowe oraz
badanie jak konfiguracja sprzętowa użytkownika wpływa na odbiór i jakość
symulacji.

\item Podsumowanie, wnioski, oraz pomysły na dalszy rozwój aplikacji.
\end{itemize}
