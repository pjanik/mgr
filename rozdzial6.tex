\chapter{Podsumowanie oraz dalszy rozwój}

Celem niniejszej pracy było stworzenie interaktywnej, edukacyjnej symulacji,
która modeluje mechanizmy przekazywania ciepła takie jak przewodnictwo cieplne
oraz konwekcja. Ponadto, możliwe miało być również modelowanie zagadnień
związanych z szeroko pojętą dynamiką płynów.

Po dogłębnej analizie oraz ocenie aplikacji \en, można stwierdzić iż udało się w
pełni zrealizować postawione cele. Powstało narzędzie o wartościowym,
edukacyjnym charakterze, realizujące wszelkie założenia projektowe. Ta
interaktywna aplikacja doskonale wpisuje się w ideę wirtualnych podręczników i
laboratoriów, umożliwiając użytkownikom łatwiejsze zrozumienie praw fizyki,
które rządzą transferem energii. W dobie szczególnej dbałości o środowisko
naturalne oraz racjonalną gospodarkę energią, dogłębne zrozumienie tego
zagadnienia staje się szczególnie istotne. Dokładniej możliwości symulatora
przybliża rozdział \ref{cha:mozliwosci}.

Także z czysto technicznego punktu widzenia udało się osiągnąć wartościowe
rezultaty. Aplikacja intensywnie wykorzystuje najnowocześniejsze standardy
technologii internetowych dostarczone przez specyfikację HTML5. Ponadto,
świadoma implementacja umożliwiła poradzenie sobie z trudnościami związanymi z
ograniczeniami środowiska przeglądarki internetowej oraz języka \js. Szczegóły
implementacyjne oraz architektoniczne porusza rozdział \ref{cha:implementacja}.

Przeprowadzona została również nowatorska optymalizacja polegająca na
przeniesieniu obliczeń fizycznych na procesor karty graficznej przy użyciu
technologii \ow{WebGL}. Przedstawione w rozdziale \ref{cha:oblGPU} opracowanie
merytoryczne tej optymalizacji oraz analiza jej wpływu na wydajność
zaprezentowana w rozdziale \ref{cha:ocena} mogą stanowić niezwykle użyteczną
pomoc przy próbie wprowadzenia podobnego udoskonalenia w innej aplikacji.

Oczywiście, wciąż istnieją pola gdzie aplikacja może i powinna być dalej
rozwijana. Na pewno interesującym udoskonaleniem byłoby lepsze wsparcie większych
siatek symulacyjnych, co powinno pozwolić uzyskać jeszcze bardziej wartościowe
oraz ciekawe wyniki symulacji. Wymaga to prawdopodobnie zmian w samych
silnikach fizycznych i stanowi doskonały cel na przyszłość.
