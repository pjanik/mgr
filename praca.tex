\documentclass[pdflatex,11pt]{aghdpl}
% \documentclass{aghdpl}               % przy kompilacji programem latex
% \documentclass[pdflatex,en]{aghdpl}  % praca w języku angielskim
\usepackage[polish]{babel}
\usepackage[utf8]{inputenc}

% dodatkowe pakiety
\usepackage{enumerate}
\usepackage{listings}
\usepackage{color}
\definecolor{lightgray}{rgb}{.96,.96,.96}
\definecolor{darkgray}{rgb}{.4,.4,.4}
\definecolor{purple}{rgb}{0.65, 0.12, 0.82}

\lstloadlanguages{TeX}

\lstset{
  literate={ą}{{\k{a}}}1
           {ć}{{\'c}}1
           {ę}{{\k{e}}}1
           {ó}{{\'o}}1
           {ń}{{\'n}}1
           {ł}{{\l{}}}1
           {ś}{{\'s}}1
           {ź}{{\'z}}1
           {ż}{{\.z}}1
           {Ą}{{\k{A}}}1
           {Ć}{{\'C}}1
           {Ę}{{\k{E}}}1
           {Ó}{{\'O}}1
           {Ń}{{\'N}}1
           {Ł}{{\L{}}}1
           {Ś}{{\'S}}1
           {Ź}{{\'Z}}1
           {Ż}{{\.Z}}1
}

\lstdefinelanguage{JavaScript}{
  keywords={typeof, new, true, false, catch, function, return, null, catch, switch, var, if, in, while, do, else, case, break},
  keywordstyle=\color{blue}\bfseries,
  ndkeywords={class, export, boolean, throw, implements, import, this},
  ndkeywordstyle=\color{darkgray}\bfseries,
  identifierstyle=\color{black},
  sensitive=false,
  comment=[l]{//},
  morecomment=[s]{/*}{*/},
  commentstyle=\color{purple}\ttfamily,
  stringstyle=\color{red}\ttfamily,
  morestring=[b]',
  morestring=[b]"
}

\lstset{
   language=JavaScript,
   backgroundcolor=\color{lightgray},
   extendedchars=true,
   basicstyle=\footnotesize\ttfamily,
   showstringspaces=false,
   showspaces=false,
   numbers=left,
   numberstyle=\footnotesize,
   numbersep=9pt,
   tabsize=4,
   breaklines=true,
   showtabs=false,
   captionpos=b
}
%---------------------------------------------------------------------------

\author{Piotr Janik}
\shortauthor{P. Janik}

\titlePL{Wydajna symulacja dynamiki płynów i~przewodnictwa cieplnego w~środowisku przeglądarki internetowej}
\titleEN{Efficient Simulation of Fluid Dynamics and Heat Transfer in~the~Web~Browser Environment}

\shorttitlePL{Wydajna symulacja dynamiki płynów i przewodnictwa cieplnego w przeglądarce internetowej} % skrócona wersja tytułu jeśli jest bardzo długi
\shorttitleEN{Efficient Simulation of Fluid Dynamics and Heat Transfer in the Web Browser Environment}

\thesistypePL{Praca magisterska}
\thesistypeEN{Master of Science Thesis}

\supervisorPL{prof. dr hab. inż. Witold Dzwinel}
\supervisorEN{Witold Dzwinel, Prof.}

\date{2012}

\departmentPL{Katedra Informatyki}
\departmentEN{Department of Computer Science}

\facultyPL{Wydział Elektrotechniki, Automatyki, Informatyki i Elektroniki}
\facultyEN{Faculty of Electrical Engineering, Automatics, Computer Science and Electronics}

\acknowledgements{Serdecznie dziękuję \dots}



\setlength{\cftsecnumwidth}{10mm}

%---------------------------------------------------------------------------

\begin{document}

\titlepages

\begin{abstract}

Celem niniejszej pracy było stworzenie wydajnej symulacji dynamiki płynów oraz
przewodnictwa cieplnego działającej w środowisku przeglądarki internetowej.
Aplikacja ma charakter edukacyjny, a jej głównym zadaniem jest wspieranie
użytkownika w dogłębnym zrozumieniu procesu transferu energii, w tym zjawisk
takich jak przewodnictwo cieplne, konwekcja czy różnorodne przepływy gazów i
cieczy. Projekt został zrealizowany przy współpracy \mbox{z The Concord
Consortium}, amerykańską organizacją \mbox{non-profit} zajmującą się wspieraniem
edukacji poprzez technologię.

Implementacja symulatora w przeglądarce internetowej była niezwykle istotna ze
względu na jego edukacyjne zastosowanie -- kluczowym wymaganiem była dostępność
dla jak najszerszego grona użytkowników, przenośność, wieloplatformowość oraz
możliwość łatwego osadzania w wirtualnych podręcznikach szkolnych nowej
generacji. Z drugiej strony, o jakości symulacji fizycznej w dużej mierze
decyduje jej wydajność czyli efektywne wykorzystanie zasobów. Do niedawna stało
to w sprzeczności z implementacją w języku \emph{JavaScript} oraz wykonywaniem w
środowisku przeglądarki internetowej. 
 
Realizacja tych pozornie wykluczających się wymagań została osiągnięta dzięki
implementacji uwzględniającej budowę i ograniczenia silników \emph{JavaScript} w
nowoczesnych przeglądarkach oraz dzięki przeniesieniu większości obliczeń na
procesor karty graficznej wykorzystując technologię \emph{WebGL}. Jest to
podejście nowatorskie, gdyż dopiero wraz z niedawnym rozpowszechnieniem się
standardu \emph{HTML5} zasoby kart graficznych stały się dostępne dla aplikacji
internetowych w tak szerokim zakresie.

W pracy zostały przedstawione najważniejsze, nowoczesne techniki umożliwiające
tworzenie wydajnych, równoległych aplikacji działających w przeglądarce
internetowej z wykorzystaniem języka \emph{JavaScript} oraz technologii
\emph{WebGL}. Szczegółowo zostały również opracowane rezultaty oraz korzyści
płynące ze zrównoleglenia symulacji fizycznej. W~\mbox{efekcie} powstał wydajny
symulator, który dzięki swojej dostępności oraz jakości może mieć niezwykle
szeroki wpływ na edukację i zrozumienie fizyki przez użytkowników na różnych
etapach procesu kształcenia.

\end{abstract}


\tableofcontents
\clearpage

\chapter{Wprowadzenie}
	\section{Motywacja}
	\section{Cel pracy}
	\section{Organizacja dokumentu}

\chapter{Wprowadzenie}
\label{cha:wprowadzenie}

Większość ludzi w swoich domach i pracy może korzystać z dorobku technologicznej rewolucji, która
miała miejsce w ostatnich latach. Jednak w niektórych dziedzinach życia zmiany następują znacznie
wolniej - jedną z nich jest edukacja. Komputery i internet stały się dostępne w większości szkół,
jednak programy i metody nauczania często nie nadążają za postępem technologicznym, są wciąż
reliktem poprzedniej epoki. Uczniowie pracują na komputerach najnowszej generacji, jednak zwykle
wykorzystują tylko ułamki ich możliwości, nie mając dostępu do narzędzi, które faktycznie mogłyby
przyczynić się do lepszego zrozumienia poruszanych na lekcjach zagadnień. Najczęściej komputer i
internet stają się po prostu źródłem łatwo dostępnej wiedzy czy też miejscem gdzie pewne problemy
można próbować rozwiązywać wspólnie. Jest to oczywiście prawidłowe i wartościowe wykorzystanie
nowoczesnej technologi, ale jednocześnie też dosyć powierzchowne i niewyczerpujące jej pełnych
możliwości. Problemy i wyzwania stojące przed współczesną edukacją szerzej porusza Andrew A. Zucker
\cite{Zuc2009}.

Edukacja może skorzystać na technologicznej rewolucji w znacznie większym stopniu - jednym z
pomysłów są wirtualne laboratoria, które pozwolą uczniom eksplorować wybrane zagadnienia w sposób
interaktywny, szczególnie podczas nauczania przedmiotów ścisłych i przyrodniczych, tak istotnych w
dzisiejszych czasach. Cyfryzacja powinna zmienić tradycyjne oblicze lekcji z podręcznikiem i
zeszytem na pracę przy narzędziach edukacyjnych nowej generacji, wykorzystując powszechną dostępność
nowoczesnych technologii. Takie aplikacje również doskonale wpisują się w popularną ideę cyfrowych
podręczników. Dosłowne przeniesienie zawartości papierowych książek na ekrany komputerów nie
wiązałoby się ze znaczącymi zmianami - inny byłby tylko nośnik słów, wiedzy. Bez zmian natomiast
pozostałby sam proces i metody uczenia przez uczniów i studentów. Jednak jeśli wirtualny podręcznik
zostanie zintegrowany z interaktywnymi aplikacjami, pozwoli to zupełnie zmienić oblicze nauki. Uczeń
będzie miał możliwość prawdziwej eksploracji zagadnień, eksperymentowania we własnym domu, przed
własnym komputerem, podczas codziennej nauki, która może zamienić się w prawdziwą, wartościową i
przede wszystkim rozwijającą przygodę.

Najlepszym pomysłem dla podręczników przyszłości wydaje się umiejscowienie ich w internecie. Dzięki
dystrybucji poprzez to medium można uzyskać niezwykle łatwy i powszechny dostęp, jako że połączenie
z internetem jest w dzisiejszych czasach czymś w pełni osiągalnym. Internetowa dystrybucja niesie
również mnóstwo korzyści nie tylko dla użytkowników podręczników, ale także dla ich twórców -
wystarczy wymienić zalety takie jak łatwość aktualizacji i docierania do odbiorców. W związku z tym,
również narzędzia stanowiące interaktywne elementy podręczników przyszłości powinny być
przystosowane do działania w środowisku przeglądarki internetowej. Jest to zadanie wymagające,
jednak niedawny rozwój technologii i standardów internetowych takich jak HTML5 oraz WebGL, jak
również gwałtowne zmiany w samych przeglądarkach internetowych, dają ogromne możliwości w tej
materii.

Wymienione pomysły nie są tylko planami na przyszłość - te zmiany już powoli następują, cyfrowe
podręczniki i wirtualne laboratoria są trakcie rozwoju. Jedną z organizacji zajmujących się
wprowadzaniem najnowszych osiągnięć techniki do szkół jest \mbox{The Concord Consortium}. 

\section{Cel pracy}
\label{sec:celPracy}

W wyniku współpracy ze wspomnianą organizacją \mbox{The Concord Consortium} powstał wydajny
symulator fizyczny prezentujący zjawisko przewodnictwa cieplnego oraz dynamikę płynów działający w
środowisku przeglądarki internetowej. Ta interaktywna aplikacja doskonale wpisuje się w
przedstawioną ideę wirtualnych podręczników i laboratoriów, umożliwiając użytkownikom łatwiejsze
zrozumienie praw fizyki, które rządzą transferem energii.

Celem niniejszej pracy jest przedstawienie rozwiązań, które umożliwiły powstanie symulatora, ze
szczególnym naciskiem na technologię WebGL, której niestandardowe i nowatorskie zastosowanie
pozwoliło zrównoleglić obliczenia fizyczne i znaczący wzrost wydajności.

\section{Motywacja}
\label{sec:motywacja}


\section{Organizacja dokumentu}
\label{sec:organizacjaDokumentu}

[TODO: tymczasowe, poprawić]

Pierwszym rozdziałem jest z niniejszy wstęp, natomiast następne przedstawiają kolejno:

\begin{itemize}

\item Wprowadzenie do problematyki symulatora dynamiki płynów i przewodnictwa cieplnego, jego
znaczenie dla edukacji oraz przegląd istniejących, podobnych rozwiązań.

\item Opis implementacji aplikacji, ze szczególnym uwzględnieniem architektury.

\item Przedstawienie sposobu w jaki silniki fizyczne zostały zrównoleglone przy użyciu technologii
\mbox{WebGL}.

\item Ocenę systemu, w szczególności testy jakościowe, wydajnościowe oraz badanie jak konfiguracja
sprzętowa użytkownika wpływa na odbiór i jakość symulacji.

\item Podsumowanie, wnioski, oraz pomysły na dalszy rozwój aplikacji.

\end{itemize}

	
	
\chapter{Symulacja dynamiki płynów i przewodnictwa cieplnego jako wartościowe narzędzie edukacyjne}
	\section{Zastosowanie oraz wpływ na edukację}
	\section{Motywacja i uzasadnienie osadzenia symulacji w środowisku przeglądarki internetowej}
	\section{Zastosowane silniki fizyczne}
		\subsection{Równanie Naviera-Stokesa}
		\subsection{Równanie przewodnictwa cieplnego}
	\section{Zjawiska modelowane przez symulator}
		\subsection{Transfer ciepła}
			\subsubsection{Przewodnictwo cieplne}
			\subsubsection{Konwekcja}
		\subsection{Przepływ płynów}
			\subsubsection{Przepływ laminarny}
			\subsubsection{Przepływ turbulentny}
	\section{Dostępne, istniejące rozwiązania}
		\subsection{Przegląd}
		\subsection{Prekursor systemu - aplikacja Energy2D}


\chapter{Problematyka tworzenia złożonych systemów w języku JavaScript}
	\section{Główne braki środowiska JavaScript w kontekście złożonych systemów}
	
	\section{Technologie przełamujące ograniczenia JavaScript}
		\subsection{Nowoczesne przeglądarki internetowe oraz rozwój interpreterów JavaScript}
			\subsubsection{Chromium V8}
		\subsection{HTML5}
		\subsection{WebGL}
		\subsection{Specyfikacja CommonJS}
		\subsection{Node.js jako zaawansowany interpreter JavaScript oderwany od przeglądarki}
			\subsubsection{Przegląd możliwości}
			\subsubsection{Schemat tworzenia aplikacji działającej zarówno w przeglądarce jak i
							środowisku Node.js}
									
	\section{Techniki zrównoleglania aplikacji JavaScript}
		\subsection{Typowe metody, a przeglądarka internetowa}
		\subsection{WebGL -- otwarcie dostępu do zasobów karty graficznej}
		\subsection{Niskopoziomowe przenoszenie obliczeń na procesor karty graficznej}		

\chapter{Implementacja symulatora w środowisku przeglądarki internetowej}
	\section{Architektura - wzorzec Model-View-Controller}
	\section{Zgodność ze środowiskiem Node.js oraz przeglądarką internetową}
	\section{Przegląd najistotniejszych jednostek symulatora}
	\section{Optymalizacje pod kątem współczesnych interpreterów JavaScript}

\chapter{Przeniesienie obliczeń fizycznych na procesor karty graficznej}

Niniejszy rozdział prezentuje techniki zastosowane w celu przeniesienia
głównych obliczeń fizycznych na kartę graficzną. Na początku opisane są
tradycyjne podejścia do tego problemu dla aplikacji działających w natywnym
środowisku systemu operacyjnego oraz ich odniesienie do środowiska oferowanego
przez przeglądarki internetowe. Następnie zaprezentowana jest technologia
\mbox{WebGL} dzięki której dokonano zrównoleglenia obliczeń w przypadku
symulatora fizycznego będącego przedmiotem opracowania niniejszej pracy. W
ostatnim podrozdziale opisane zostały najważniejsze szczegóły implementacyjne,
które mogą być niezwykle pomocne przy próbach podobnej optymalizacji innych
aplikacji.

Dzięki przeniesieniu obliczeń na GPU uzyskano niezwykle istotny wzrost
wydajności. Dokładna analiza zysków ze zrównoleglenia symulacji jest tematyką
kolejnego rozdziału [TODO: podać dokładny rozdział].

\section{Typowe metody przenoszenia obliczeń na GPU, a przeglądarka internetowa}

Współczesna karty graficzne posiadają ogromną moc obliczeniową -- wielokrotnie
większą od centralnego procesora przy założeniu, że obliczenia da się
wykonywać w sposób równoległy. Pomysł, aby przenieść część obliczeń ogólnego
zastosowania na kartę graficzną pojawił się wraz z dynamicznym rozwojem
procesorów graficznych. Szczególnie istotnym momentem było wprowadzenie
programowalnych jednostek cieniujących (specyfikacja \emph{DirectX 8}). Dalszy
rozwój obliczeń ogólnego zastosowania na kartach graficznych (ang. \emph
{General-purpose computing on graphics processing units}, w skrócie GPGPU)
miał miejsce wraz z wprowadzeniem technologii, które ukryły złożoność dostępu
do zasobów karty graficznej i udostępniły interfejs wysokiego poziomu.
Wiodącymi technologiami tego typu są \emph{OpenCL} (rozwiązanie otwarte) oraz
\emph{CUDA} (zamknięte rozwiązanie firmy NVIDIA, działające wyłącznie na
sprzęcie tego producenta).

\subsection{Niskopoziomowe programowanie jednostek cieniujących}

Jest to najstarsze podejście do przeprowadzania obliczeń na procesorze karty
graficznej. Programista w swoisty sposób ,,oszukuje'' kartę graficzną,
przeprowadzając renderowanie prostej geometrii wyłącznie w celu uruchomienia
własnych programów jednostek cieniujących, które wykonują obliczenia często
nie mające nic wspólnego z generowaniem obrazu.

Programowanie tego typu bywa wymagające. Łatwo popełnić błędy, a programista
musi mieć przynajmniej podstawową wiedzę o programowaniu grafiki 3D.

\subsection{Technologie wyższego poziomu}

Technologie, które przyczyniły się do gwałtownego wzrostu popularności
obliczeń na kartach graficznych to szczególnie \emph{OpenCL} oraz \emph{CUDA}.
Udostępniają  one znacznie wyższy poziom abstrakcji -- programista jest
zwolniony z obowiązku przyswojenia sobie niskopoziomowych mechanizmów
rządzących działaniem kart graficznych (choć ta wiedza pozwala tworzyć
aplikacje efektywniejsze).

\subsection{Środowisko przeglądarka internetowa}

Opisane wcześniej technologie dotyczą aplikacji pisanych w natywnym środowisku
systemu operacyjnego. Aby programować jednostki cieniujące wystarczy
podstawowy dostęp do standardowego interfejsu OpenGL bądź Direct3D.
Implementacje tych interfejsów można znaleźć dla prawie każdego współczesnego
języka programowania. Technologie wyższego poziomu takie jak OpenCL oraz CUDA
również posiadają implementacje w wielu różnych językach, choć najczęstszym
środowiskiem ich działania są zwykle aplikacje napisane w C bądź C++.

Przeglądarka internetowa udostępnia programiście JavaScript środowisko bardzo
ograniczone ze względów bezpieczeństwa. Technologie programowania kart
graficznych wyższego poziomu nie są (jeszcze) dostępne. Aktualnie trwają
intensywne prace nad implementacją standardu OpenCL w przeglądarce
internetowej - WebCL. Jednak technologia ta w aktualnym momencie jest na
bardzo wczesnym etapie rozwoju. Więcej informacji można znaleźć na stronie
internetowej: http://www.khronos.org/webcl/.

Jednak niedawno, wraz z nadejściem standardu HTML5, został dodany podstawowy
dostęp do zasobów karty graficznej. Pojawiła się możliwość programowania
jednostek cieniujących kart graficznych i tym samym przeprowadzania na nich
obliczeń ogólnego zastosowania. Zadania te są realizowane przy użyciu
technologii WebGL, która jest implementacją standardu OpenGL ES 2.0.

\section{WebGL -- otwarcie dostępu do zasobów karty graficznej w przeglądarce}

\section{Opis najważniejszych zagadnień implementacji silników fizycznych przy użyciu WebGL}



	
\chapter{Ocena systemu}
	\section{Realizacja kluczowych wymagań}
	\section{Ocena dostępności aplikacji dla potencjalnych użytkowników}
		\subsection{Wpływ posiadanej konfiguracji sprzętowej i oprogramowania na symulator}
	\section{Modelowanie wybranych zjawisk fizycznych jako testy jakościowe symulatora}
	\section{Testy wydajnościowe}
		\subsection{Porównanie wydajności z symulatorem Energy2D opartym o platformę Java}
		\subsection{Zysk wydajności wynikający z przeniesienia obliczeń na GPU}
	\section{Wpływ optymalizacji i równoległości na jakość symulacji}
	\section{Podsumowanie oceny}
	
\chapter{Wnioski}
	\section{Podsumowanie}
	\section{Dalszy rozwój}

%\chapter{Wprowadzenie}
\label{cha:wprowadzenie}

Większość ludzi w swoich domach i pracy może korzystać z dorobku technologicznej rewolucji, która
miała miejsce w ostatnich latach. Jednak w niektórych dziedzinach życia zmiany następują znacznie
wolniej - jedną z nich jest edukacja. Komputery i internet stały się dostępne w większości szkół,
jednak programy i metody nauczania często nie nadążają za postępem technologicznym, są wciąż
reliktem poprzedniej epoki. Uczniowie pracują na komputerach najnowszej generacji, jednak zwykle
wykorzystują tylko ułamki ich możliwości, nie mając dostępu do narzędzi, które faktycznie mogłyby
przyczynić się do lepszego zrozumienia poruszanych na lekcjach zagadnień. Najczęściej komputer i
internet stają się po prostu źródłem łatwo dostępnej wiedzy czy też miejscem gdzie pewne problemy
można próbować rozwiązywać wspólnie. Jest to oczywiście prawidłowe i wartościowe wykorzystanie
nowoczesnej technologi, ale jednocześnie też dosyć powierzchowne i niewyczerpujące jej pełnych
możliwości. Problemy i wyzwania stojące przed współczesną edukacją szerzej porusza Andrew A. Zucker
\cite{Zuc2009}.

Edukacja może skorzystać na technologicznej rewolucji w znacznie większym stopniu - jednym z
pomysłów są wirtualne laboratoria, które pozwolą uczniom eksplorować wybrane zagadnienia w sposób
interaktywny, szczególnie podczas nauczania przedmiotów ścisłych i przyrodniczych, tak istotnych w
dzisiejszych czasach. Cyfryzacja powinna zmienić tradycyjne oblicze lekcji z podręcznikiem i
zeszytem na pracę przy narzędziach edukacyjnych nowej generacji, wykorzystując powszechną dostępność
nowoczesnych technologii. Takie aplikacje również doskonale wpisują się w popularną ideę cyfrowych
podręczników. Dosłowne przeniesienie zawartości papierowych książek na ekrany komputerów nie
wiązałoby się ze znaczącymi zmianami - inny byłby tylko nośnik słów, wiedzy. Bez zmian natomiast
pozostałby sam proces i metody uczenia przez uczniów i studentów. Jednak jeśli wirtualny podręcznik
zostanie zintegrowany z interaktywnymi aplikacjami, pozwoli to zupełnie zmienić oblicze nauki. Uczeń
będzie miał możliwość prawdziwej eksploracji zagadnień, eksperymentowania we własnym domu, przed
własnym komputerem, podczas codziennej nauki, która może zamienić się w prawdziwą, wartościową i
przede wszystkim rozwijającą przygodę.

Najlepszym pomysłem dla podręczników przyszłości wydaje się umiejscowienie ich w internecie. Dzięki
dystrybucji poprzez to medium można uzyskać niezwykle łatwy i powszechny dostęp, jako że połączenie
z internetem jest w dzisiejszych czasach czymś w pełni osiągalnym. Internetowa dystrybucja niesie
również mnóstwo korzyści nie tylko dla użytkowników podręczników, ale także dla ich twórców -
wystarczy wymienić zalety takie jak łatwość aktualizacji i docierania do odbiorców. W związku z tym,
również narzędzia stanowiące interaktywne elementy podręczników przyszłości powinny być
przystosowane do działania w środowisku przeglądarki internetowej. Jest to zadanie wymagające,
jednak niedawny rozwój technologii i standardów internetowych takich jak HTML5 oraz WebGL, jak
również gwałtowne zmiany w samych przeglądarkach internetowych, dają ogromne możliwości w tej
materii.

Wymienione pomysły nie są tylko planami na przyszłość - te zmiany już powoli następują, cyfrowe
podręczniki i wirtualne laboratoria są trakcie rozwoju. Jedną z organizacji zajmujących się
wprowadzaniem najnowszych osiągnięć techniki do szkół jest \mbox{The Concord Consortium}. 

\section{Cel pracy}
\label{sec:celPracy}

W wyniku współpracy ze wspomnianą organizacją \mbox{The Concord Consortium} powstał wydajny
symulator fizyczny prezentujący zjawisko przewodnictwa cieplnego oraz dynamikę płynów działający w
środowisku przeglądarki internetowej. Ta interaktywna aplikacja doskonale wpisuje się w
przedstawioną ideę wirtualnych podręczników i laboratoriów, umożliwiając użytkownikom łatwiejsze
zrozumienie praw fizyki, które rządzą transferem energii.

Celem niniejszej pracy jest przedstawienie rozwiązań, które umożliwiły powstanie symulatora, ze
szczególnym naciskiem na technologię WebGL, której niestandardowe i nowatorskie zastosowanie
pozwoliło zrównoleglić obliczenia fizyczne i znaczący wzrost wydajności.

\section{Motywacja}
\label{sec:motywacja}


\section{Organizacja dokumentu}
\label{sec:organizacjaDokumentu}

[TODO: tymczasowe, poprawić]

Pierwszym rozdziałem jest z niniejszy wstęp, natomiast następne przedstawiają kolejno:

\begin{itemize}

\item Wprowadzenie do problematyki symulatora dynamiki płynów i przewodnictwa cieplnego, jego
znaczenie dla edukacji oraz przegląd istniejących, podobnych rozwiązań.

\item Opis implementacji aplikacji, ze szczególnym uwzględnieniem architektury.

\item Przedstawienie sposobu w jaki silniki fizyczne zostały zrównoleglone przy użyciu technologii
\mbox{WebGL}.

\item Ocenę systemu, w szczególności testy jakościowe, wydajnościowe oraz badanie jak konfiguracja
sprzętowa użytkownika wpływa na odbiór i jakość symulacji.

\item Podsumowanie, wnioski, oraz pomysły na dalszy rozwój aplikacji.

\end{itemize}


%\chapter{Przeniesienie obliczeń fizycznych na procesor karty graficznej}

Niniejszy rozdział prezentuje techniki zastosowane w celu przeniesienia
głównych obliczeń fizycznych na kartę graficzną. Na początku opisane są
tradycyjne podejścia do tego problemu dla aplikacji działających w natywnym
środowisku systemu operacyjnego oraz ich odniesienie do środowiska oferowanego
przez przeglądarki internetowe. Następnie zaprezentowana jest technologia
\mbox{WebGL} dzięki której dokonano zrównoleglenia obliczeń w przypadku
symulatora fizycznego będącego przedmiotem opracowania niniejszej pracy. W
ostatnim podrozdziale opisane zostały najważniejsze szczegóły implementacyjne,
które mogą być niezwykle pomocne przy próbach podobnej optymalizacji innych
aplikacji.

Dzięki przeniesieniu obliczeń na GPU uzyskano niezwykle istotny wzrost
wydajności. Dokładna analiza zysków ze zrównoleglenia symulacji jest tematyką
kolejnego rozdziału [TODO: podać dokładny rozdział].

\section{Typowe metody przenoszenia obliczeń na GPU, a przeglądarka internetowa}

Współczesna karty graficzne posiadają ogromną moc obliczeniową -- wielokrotnie
większą od centralnego procesora przy założeniu, że obliczenia da się
wykonywać w sposób równoległy. Pomysł, aby przenieść część obliczeń ogólnego
zastosowania na kartę graficzną pojawił się wraz z dynamicznym rozwojem
procesorów graficznych. Szczególnie istotnym momentem było wprowadzenie
programowalnych jednostek cieniujących (specyfikacja \emph{DirectX 8}). Dalszy
rozwój obliczeń ogólnego zastosowania na kartach graficznych (ang. \emph
{General-purpose computing on graphics processing units}, w skrócie GPGPU)
miał miejsce wraz z wprowadzeniem technologii, które ukryły złożoność dostępu
do zasobów karty graficznej i udostępniły interfejs wysokiego poziomu.
Wiodącymi technologiami tego typu są \emph{OpenCL} (rozwiązanie otwarte) oraz
\emph{CUDA} (zamknięte rozwiązanie firmy NVIDIA, działające wyłącznie na
sprzęcie tego producenta).

\subsection{Niskopoziomowe programowanie jednostek cieniujących}

Jest to najstarsze podejście do przeprowadzania obliczeń na procesorze karty
graficznej. Programista w swoisty sposób ,,oszukuje'' kartę graficzną,
przeprowadzając renderowanie prostej geometrii wyłącznie w celu uruchomienia
własnych programów jednostek cieniujących, które wykonują obliczenia często
nie mające nic wspólnego z generowaniem obrazu.

Programowanie tego typu bywa wymagające. Łatwo popełnić błędy, a programista
musi mieć przynajmniej podstawową wiedzę o programowaniu grafiki 3D.

\subsection{Technologie wyższego poziomu}

Technologie, które przyczyniły się do gwałtownego wzrostu popularności
obliczeń na kartach graficznych to szczególnie \emph{OpenCL} oraz \emph{CUDA}.
Udostępniają  one znacznie wyższy poziom abstrakcji -- programista jest
zwolniony z obowiązku przyswojenia sobie niskopoziomowych mechanizmów
rządzących działaniem kart graficznych (choć ta wiedza pozwala tworzyć
aplikacje efektywniejsze).

\subsection{Środowisko przeglądarka internetowa}

Opisane wcześniej technologie dotyczą aplikacji pisanych w natywnym środowisku
systemu operacyjnego. Aby programować jednostki cieniujące wystarczy
podstawowy dostęp do standardowego interfejsu OpenGL bądź Direct3D.
Implementacje tych interfejsów można znaleźć dla prawie każdego współczesnego
języka programowania. Technologie wyższego poziomu takie jak OpenCL oraz CUDA
również posiadają implementacje w wielu różnych językach, choć najczęstszym
środowiskiem ich działania są zwykle aplikacje napisane w C bądź C++.

Przeglądarka internetowa udostępnia programiście JavaScript środowisko bardzo
ograniczone ze względów bezpieczeństwa. Technologie programowania kart
graficznych wyższego poziomu nie są (jeszcze) dostępne. Aktualnie trwają
intensywne prace nad implementacją standardu OpenCL w przeglądarce
internetowej - WebCL. Jednak technologia ta w aktualnym momencie jest na
bardzo wczesnym etapie rozwoju. Więcej informacji można znaleźć na stronie
internetowej: http://www.khronos.org/webcl/.

Jednak niedawno, wraz z nadejściem standardu HTML5, został dodany podstawowy
dostęp do zasobów karty graficznej. Pojawiła się możliwość programowania
jednostek cieniujących kart graficznych i tym samym przeprowadzania na nich
obliczeń ogólnego zastosowania. Zadania te są realizowane przy użyciu
technologii WebGL, która jest implementacją standardu OpenGL ES 2.0.

\section{WebGL -- otwarcie dostępu do zasobów karty graficznej w przeglądarce}

\section{Opis najważniejszych zagadnień implementacji silników fizycznych przy użyciu WebGL}





% itd.
% \appendix
% \include{dodatekA}
% \include{dodatekB}
% itd.

\bibliographystyle{alpha}
\bibliography{bibliografia}
%\begin{thebibliography}{1}
%
%\bibitem{Dil00}
%A.~Diller.
%\newblock {\em LaTeX wiersz po wierszu}.
%\newblock Wydawnictwo Helion, Gliwice, 2000.
%
%\bibitem{Lam92}
%L.~Lamport.
%\newblock {\em LaTeX system przygotowywania dokumentów}.
%\newblock Wydawnictwo Ariel, Krakow, 1992.
%
%\bibitem{Alvis2011}
%M.~Szpyrka.
%\newblock {\em {On Line Alvis Manual}}.
%\newblock AGH University of Science and Technology, 2011.cccccc
%\newblock \\\texttt{http://fm.ia.agh.edu.pl/alvis:manual}.
%
%\end{thebibliography}

\end{document}
